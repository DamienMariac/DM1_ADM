\documentclass{article}
\usepackage{graphicx}

\title{TP1 ADM}
\author{MARIAC Damien, Matteo Scaia}
\date{\today} 

\begin{document}

\maketitle

\begin{figure}[h] 
    \centering
    \includegraphics[width=0.5\textwidth]{ssd_logo.png} 
\end{figure}

\newpage
\tableofcontents
\newpage

\section{Introduction}
On dispose d'un jeu de donnees présentant une étude de 27 espèces d'arbres dans 1000 parcelles d'une forêt.
Il s'agit d'étudier la variabilité des densités de peuplement d'espèces arborées dans différentes parcelles de la forêt du bassin du Congo.
Nous disposons dans notre jeu de donnée 30 variables quantitatives dont : 27 variables de comptage des espèces, la surface de la parcelle, 2 variables une forestier et une géologique.
Et une variable qualitative "code".

\section{Partie1}
\subsection{inertie et barycentre}

Nous cherchons à calculer la densité de peuplement de chaque espece par unité de surface. Nous calculons alors pour chaque parcelle:
\[
(d_j^i)_{1 \leq j \leq 1000}^{1 \leq i \leq 27} = \frac{x_j^i}{s_j}
\]

\begin{table}[h]
    \centering
    \caption{Extrait de densité}
    \label{tab:donnees_extrait}
    \begin{tabular}{|c|c|c|c|c|}
    \hline
    \textbf{Code} & \textbf{Gen1} & \textbf{Gen5} & \textbf{Gen10}\\
    \hline
    1 & 0 & 0 & 2.200 \\
    2 & 0.6 & 0.133 & 1.333  \\
    3 & 0.514 & 0.057 & 3.6 \\
    4 & 0 & 0.439 & 0.244 \\
    5 & 0.095 & 0 & 0.476 \\
    \hline
    \end{tabular}
    \end{table}


Nous utiliserons des densités plutôt que des comptages car cela permet de normaliser les données par rapport à la taille de la parcelle,
ce qui rend les comparaisons entre les parcelles équitables.
\\
Nous devons centrer et reduire les variables quantitatives dans le but de mieux comparer celles qui decrivent les différents densité.
Nous allons alors utiliser :
\[
(x_j^i)_{1 \leq i \leq 27} = \frac{x_j^i - \bar{x}_j}{\sigma_j}
\]

Avec $\bar{x}_j$ la moyenne pour la j eme variable et  $\sigma_j$ l'ecart-type de la variable quantitative j.
\\
\\
Par consequents on a:
\\
Barycentre à l'origine : Après centrage, la moyenne de chaque variable (densité centrée-réduite) doit être 0.
\\
Inertie totale égale à 27 : L'inertie, qui mesure la dispersion du nuage de points, est égale à la somme des variances des variables, 
qui doit être égale au nombre de variables après la réduction.









\section{Conclusion}
\end{document}
